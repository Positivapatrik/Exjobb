\documentclass[12pt]{article}
\usepackage[swedish]{babel}
\usepackage[T1]{fontenc}
\usepackage{lmodern}
\usepackage[utf8]{inputenc}
\usepackage{lscape}
\usepackage[table]{xcolor}
\usepackage{graphicx}

% Grupphandledningsplanering
% http://www.kth.se/polopoly_fs/1.138963!/Menu/general/column-content/attachment/arbetsgang.pdf
\begin{document}
\newcommand{\B}{\cellcolor{blue!25}}
\newcommand{\BB}{\cellcolor{blue!100}}
\newcommand{\R}{\cellcolor{red!25}}
\newcommand{\RR}{\cellcolor{red!100}}

\title{Exjobbstidsplanering}
\author{Patrik Berggren}
\maketitle

\begin{table}[!ht]
\setlength{\tabcolsep}{2pt}
\begin{tabular}{|l|c|c|c|c|c|c|c|c|c|c|c|c|c|c|c|c|c|c|c|c|c|c|}
\hline
Arbetsvecka & 
\rotatebox{90}{0} &
\rotatebox{90}{1} &
\rotatebox{90}{2} &
\rotatebox{90}{3} &
\rotatebox{90}{4} &
\rotatebox{90}{5} &
\rotatebox{90}{6} &
\rotatebox{90}{7} &
\rotatebox{90}{8} &
\rotatebox{90}{9} &
\rotatebox{90}{10} &
\rotatebox{90}{11} &
\rotatebox{90}{12} &
\cellcolor{black!25}&
\rotatebox{90}{13} &
\rotatebox{90}{14} &
\rotatebox{90}{15} &
\rotatebox{90}{16} &
\rotatebox{90}{17} &
\rotatebox{90}{18} &
\rotatebox{90}{19} &
\rotatebox{90}{20} \\
\hline
Veckonummer & 
\rotatebox{90}{3} &
\rotatebox{90}{4} &
\rotatebox{90}{5} &
\rotatebox{90}{6} &
\rotatebox{90}{7} &
\rotatebox{90}{8} &
\rotatebox{90}{9} &
\rotatebox{90}{10} &
\rotatebox{90}{11} &
\rotatebox{90}{12} &
\rotatebox{90}{13} &
\rotatebox{90}{14} &
\rotatebox{90}{15} &
\rotatebox{90}{16} &
\rotatebox{90}{17} &
\rotatebox{90}{18} &
\rotatebox{90}{19} &
\rotatebox{90}{20} &
\rotatebox{90}{21} & 
\rotatebox{90}{22} &
\rotatebox{90}{23} &
\rotatebox{90}{24} \\
\hline
Registrering & \B &&&&&&&&&&&&&&&&&&&&&  \\
\hline
Preliminär tidsplanering & \B &&&&&&&&&&&&&&&&&&&&&  \\
\hline
Specifikation med tidsplanering & \B & \B&\BB&&&&&&&&&&&&&&&&&&&  \\
\hline
Litteraturstudie && \B&\B& \B&\B&\B&\B&&&&&&&&&&&&&&  \\
\hline
Backgrundsdel rapport &&&\R& \R&\R&\R&\R&&&&&&&&&&&&&&  \\
\hline
PRO1 (spec. och inläsn) &&&&&&&&\BB&&&&&&&&&&&&&  \\
\hline
Förbereda/planera genomförandet &&&&&\B&&&&&&&&&&&&&&&&  \\
\hline
Implementering, tester &&&&&&\B&\B&\B&\B&\B&\B&\B&\B&&&&&&&&  \\
\hline
Metoddel rapport &&&&&&&&\R&\R&&&&&&&&&&&&  \\
\hline
Resultatdel rapport &&&&&&&&&&\R&\R&\R&\R&&&&&&&&  \\
\hline
Påsk &&&&&&&&&&&&&&\cellcolor{black!25}&&&&&&&  \\
\hline
Analysera resultat &&&&&&&&&&&&&&&\B&\B&&&&&  \\
\hline
Analysdel rapport &&&&&&&&&&&&&&&\R&\R&&&&&  \\
\hline
PRO2 (75 \% av exjobb klart) &&&&&&&&&&&&&&&&&\BB&&&&  \\
\hline
Arbete med rapport &&&&&&&&&&&&&&&&&\R&\R&\R&&  \\
\hline
Opponering förb/genomf &&&&&&&&&&&&&&&&&&&&\B&\BB&  \\
\hline
Presentation förb/genomf &&&&&&&&&&&&&&&&&&&&\B&\B&\BB  \\
\hline
Slutgiltig rapport &&&&&&&&&&&&&&&&&&&&\R&\R&\RR  \\
\hline
Avslutande av exjobb &&&&&&&&&&&&&&&&&&&&&&\BB  \\
\hline
\end{tabular}
\caption{Röd betyder rapportarbete och blått övrigt. Inlämningsmoment är markerade med mörkare färg.}
\end{table}

Detta är så att säga idealplaneringen. När det gäller presentation och opponering så kan de skifta beroende på tidsbokning, men det jag kommer eftersträva är att göra båda i de sista veckorna. Det finns även flest opponeringstillfällen i slutet av terminen så det borde gå att hitta tillfälle då.

Planeringen börjar på vecka 3, men den är angiven som arbetsvecka 0 då jag först vecka 4 kommer börja arbeta heltid med exjobbet.


\end{document}
