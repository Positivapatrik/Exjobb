\documentclass[12pt]{article}
%\usepackage[swedish]{babel}
\usepackage[T1]{fontenc}
\usepackage{lmodern}
\usepackage[utf8]{inputenc}
\usepackage{lscape}
\usepackage[table]{xcolor}
\usepackage{graphicx}
\usepackage{float}
\usepackage{placeins}

% Grupphandledningsplanering
% http://www.kth.se/polopoly_fs/1.138963!/Menu/general/column-content/attachment/arbetsgang.pdf
\begin{document}
\newcommand{\B}{\cellcolor{blue!25}}
\newcommand{\BB}{\cellcolor{blue!100}}
\newcommand{\R}{\cellcolor{red!25}}
\newcommand{\RR}{\cellcolor{red!100}}

\title{Master thesis specification \\
       \Large Efficient features for representing hand shape in
images}
\author{Patrik Berggren (pabergg@kth.se) \\
Supervisor: Hedvig Kjellström (hedvig@csc.kth.se)}
\maketitle
\section*{Overview and background}
Today, there is a growing need for smarter interactions between humans and computers.
One of those possible interactions is by doing hand gestures which requires the computer to detect and understand what hand pose the hand is doing.
Hand pose estimation can also be useful in applications where a robot needs to learn a task from human demonstrations.
Regardless, it is often desirable to detect the hand pose from a single 2D image instead of more intrusive methods such as using a glove.
In the later case it is much easier to detect handposes accurately since one can then use sensors in the glove or special markings on it to effectively track in what state each joint in the hand is in.

This master thesis will however focus on the first type.
Here, a common approach is to extract features from the image and then use those features to estimate a hand pose.
However, it is generally difficult to find a good approximation for the mapping from features to hand poses.
This is because the mapping is often many to many meaning that the same hand pose can give rise to different features and vice versa.
Furthermore, this mapping is discontinous or at least not very smooth meaning that a small change in the image features or estimated hand pose space may correspond to a big change in the other space.
Previously, Akshaya Thippur Shridatta have researched different image features' quality in this regard, in his master thesis ''Comparative analysis of visual shape features for applications to hand pose estimation''.
This master thesis will in a sense be a continuation on Akshayas work by further investigating ways to extract image features that displays good qualities when it comes to the qualities mentioned above which is termed smoothness, generativity and discriminativity.

\section*{More detailed specification}
Quite recently a new approach have been proposed and tested with good results when it comes to classification of images.
More specifically, the method presented uses predefined binary classifiers that together defines a binary vector for each image.
This binary vector is then used for the training of new classes.
The idea is that we in the case of hand pose estimation could use a similiar appraoch, but instead of binary classifiers use functions from the feature space to a real number, and we will thus get a vector of real values. 
Hopefully, this descriptor vector can then be used for regression to get the hand pose.
However, this depends on how well suited the mapping from images to descriptor vectors are in regard of smoothness, descriminativity, and generativity.
The master thesis primary focus is to investigate how this descriptor can be formed and what parameters seems to work best.

\subsection*{Limitations}
\begin{itemize}
\item
No real hand images will be used. Instead images will be generated with libHand.
\item
There are many possible image feature, but I will only use HOG features.
\item
The focus is on the descriptor vector and so the regression step may be limited to one regression method.
\end{itemize}

\section*{Goals}
The goal of the project would be to find a image descriptor that is well suited for regression.
The ultimate goal is of course to be able to use the findings to be able to estimate hand poses in real life.
This is, however, the ideal outcome, but the aim of the master thesis is at the very least to investigate how well a classeme-like approach can be used in conjunction with HOG features to form regression-freindly descriptors.

\section*{Equipment}
During the project I will work with libHand and matlab.

\section*{Litterature}
Apart from Akshayas master thesis a lot of the references in his thesis will be used. Also, papers which relate to the classeme approach for classifications will be relevant.
\begin{itemize}
\item
Akshayas master thesis ''Comparative Analysis of Visual Shape Features for Applications to Hand Pose Estimation'' and the related article ''Inferring Hand Pose: A Comparative Study of Visual Shape Features''. A few of the relevant references from Akshaya is:
\begin{itemize}
\item
Histograms of Oriented Gradients for Human Detection
\item
A Review on Vision-Based Full DOF Hand Motion Estimation
\item
Full dof tracking of a hand interacting with an object by modeling occlusions and physical constraints
\item
Spatio-Temporal Modeling of Grasping Actions
\item
Real-Time 3D Reconstruction of Hands in Interaction with Objects
\end{itemize}
\item
Classeme related papers:
\begin{itemize}
\item
Efficient Object Category Recognition Using Classemes
\item
PICODES: Learning a Compact Code for Novel-Category Recognition
\item
Meta-Class Features for Large-Scale Object Categorization on a Budget
\end{itemize}
\end{itemize}
The litterature study can be examined with the background section of the thesis.

\section*{Time plan}
\FloatBarrier

\begin{table}[H]
\setlength{\tabcolsep}{2pt}
\begin{tabular}{|l|c|c|c|c|c|c|c|c|c|c|c|c|c|c|c|c|c|c|c|c|c|c|}
\hline
Work week  & 
\rotatebox{90}{0} &
\rotatebox{90}{1} &
\rotatebox{90}{2} &
\rotatebox{90}{3} &
\rotatebox{90}{4} &
\rotatebox{90}{5} &
\rotatebox{90}{6} &
\rotatebox{90}{7} &
\rotatebox{90}{8} &
\rotatebox{90}{9} &
\rotatebox{90}{10} &
\rotatebox{90}{11} &
\rotatebox{90}{12} &
\cellcolor{black!25}&
\rotatebox{90}{13} &
\rotatebox{90}{14} &
\rotatebox{90}{15} &
\rotatebox{90}{16} &
\rotatebox{90}{17} &
\rotatebox{90}{18} &
\rotatebox{90}{19} &
\rotatebox{90}{20} \\
\hline
Week nr & 
\rotatebox{90}{3} &
\rotatebox{90}{4} &
\rotatebox{90}{5} &
\rotatebox{90}{6} &
\rotatebox{90}{7} &
\rotatebox{90}{8} &
\rotatebox{90}{9} &
\rotatebox{90}{10} &
\rotatebox{90}{11} &
\rotatebox{90}{12} &
\rotatebox{90}{13} &
\rotatebox{90}{14} &
\rotatebox{90}{15} &
\rotatebox{90}{16} &
\rotatebox{90}{17} &
\rotatebox{90}{18} &
\rotatebox{90}{19} &
\rotatebox{90}{20} &
\rotatebox{90}{21} & 
\rotatebox{90}{22} &
\rotatebox{90}{23} &
\rotatebox{90}{24} \\
\hline
Registration & \B &&&&&&&&&&&&&&&&&&&&&  \\
\hline
Preliminar time plan & \B &&&&&&&&&&&&&&&&&&&&&  \\
\hline
Specification with time plan & \B & \B&\B&\BB&&&&&&&&&&&&&&&&&&  \\
\hline
Study litterature && \B&\B& \B&\B&\B&\B&&&&&&&&&&&&&&  \\
\hline
Background section in thesis &&&\R& \R&\R&\R&\R&&&&&&&&&&&&&&  \\
\hline
PRO1 (7.5 hp)&&&&&&&&\BB&&&&&&&&&&&&&  \\
\hline
Prepare and plan implementation/tests &&&&&&&\B&&&&&&&&&&&&&&  \\
\hline
Implementation, tests &&&&&&&&\B&\B&\B&\B&\B&\B&&&&&&&&  \\
\hline
Method section in thesis &&&&&&&&\R&\R&&&&&&&&&&&&  \\
\hline
Result section in thesis &&&&&&&&&&\R&\R&\R&\R&&&&&&&&  \\
\hline
No work &&&&&&&&&&&&&&\cellcolor{black!25}&&&&&&&  \\
\hline
Analyse results &&&&&&&&&&&&&&&\B&\B&&&&&  \\
\hline
Analyse section in thesis &&&&&&&&&&&&&&&\R&\R&&&&&  \\
\hline
PRO2 (15 hp) &&&&&&&&&&&&&&&&&\BB&&&&  \\
\hline
Work with thesis &&&&&&&&&&&&&&&&&\R&\R&\R&&  \\
\hline
Opposition &&&&&&&&&&&&&&&&&&&&\B&\BB&  \\
\hline
Presentation &&&&&&&&&&&&&&&&&&&&\B&\B&\BB  \\
\hline
Final thesis &&&&&&&&&&&&&&&&&&&&\R&\R&\RR  \\
\hline
End of master thesis registration etc &&&&&&&&&&&&&&&&&&&&&&\BB  \\
\hline
\end{tabular}
\caption{Red means work with thesis and blue means other work. A darker colour indicates that something should be submitted for registration/examination.}
\end{table}

Both presentation and opposition depends on when there is time available, but there should be time slots for both at the end of the semester, but they might nevertheless be shifted a week or two.

The time plan begins at week 3, but it is work week 0 since full time work started week 4.

\end{document}
